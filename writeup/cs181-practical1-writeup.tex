\documentclass[11pt]{amsart}
\usepackage{geometry}                % See geometry.pdf to learn the layout options. There are lots.
\geometry{letterpaper}                   % ... or a4paper or a5paper or ... 
%\geometry{landscape}                % Activate for for rotated page geometry
%\usepackage[parfill]{parskip}    % Activate to begin paragraphs with an empty line rather than an indent
\usepackage{graphicx}
\usepackage{amssymb}
\usepackage{epstopdf}
\DeclareGraphicsRule{.tif}{png}{.png}{`convert #1 `dirname #1`/`basename #1 .tif`.png}

% Declare commands
\newcommand{\mat}[1]{\mathbf{#1}}

\title{CS 181 -- Practical 1}
\author{Casey Grun, Sam Kim, Rhed Shi}
%\date{}                                           % Activate to display a given date or no date

\begin{document}
\maketitle

\section{Warmup}

\section{Approaches considered}
K-Means, K-nearest neighbor, PCA, SVD, PMF

\section{Probabilistic Matrix Factorization}
\subsection{Derivation}
The matrix factorization approach attempts to decompose the ratings into a combination of the features of users and the features of the books. That is, the matrix $\mat{R}$ is factored into two matrices $\mat{P}$ representing the users and $\mat{Q}$ representing the books:
$$\mat{R} = \mat{P} \times \mat{Q}^T$$
where:
$$ 
\mat{P} = 
\begin{pmatrix}
	\vec{p_1} \\
	\vec{p_2} \\
	\vdots    \\
	\vec{p_N} 
\end{pmatrix} = 
\begin{pmatrix}
	p_{1,1} & p_{1,2} & \ldots & p_{1,K} \\
	p_{2,1} & p_{2,2} & \ldots & p_{2,K} \\
	\vdots  & \vdots  & \ddots & \vdots  \\
	p_{N,1} & p_{N,2} & \ldots & p_{N,K} \\
\end{pmatrix}
$$
$$ 
\mat{Q} = 
\begin{pmatrix}
	\vec{q_1} \\
	\vec{q_2} \\
	\vdots    \\
	\vec{q_N} 
\end{pmatrix} = 
\begin{pmatrix}
	q_{1,1} & q_{1,2} & \ldots & q_{1,K} \\
	q_{2,1} & q_{2,2} & \ldots & q_{2,K} \\
	\vdots  & \vdots  & \ddots & \vdots  \\
	q_{N,1} & q_{N,2} & \ldots & q_{N,K} \\
\end{pmatrix}
$$

However, we only know a subset of the full matrix $\{r_{i,j}\}_{i=1, j=1}^{N,D}$; call that subset $T = \{(i, j, r_{i,j})\}$. If we can produce the matrices $\mat{P}$ and $\mat{Q}$, we can predict any entry $r_{i,j}$ in $\mat{R}$ by taking inner products of the vectors $\vec{p_i}$ and $\vec{q_j}$:
$$\hat{r}_{i,j} = \vec{p_i} \cdot \vec{q_j}^T \approx r_{i,j}$$

However, we recognize that some aspects of the ratings for a particular book are not explained jointly by the user and book, but by either the user or the book alone; we will call these terms \emph{biases}. A bias is the component of a user's ratings (or a book's ratings) varying from the global mean, $\mu = \frac{1}{ND} \sum_{i=1}^{N}\sum_{j=1}^{D} r_{i,j}$. Let $\vec{b} = (b_1, b_2, \ldots b_N)$ represent the biases for each users and $\vec{c} = (c_1, c_2, \ldots c_D)$ represent the biases for each book. We will adopt a slightly more sophisticated model for predicting the ratings, such that:
$$\hat{r}_{i,j} = \mu + b_i + c_j + \vec{p_i} \cdot \vec{q_j}^T \approx r_{i,j}$$
or:
$$\mat{R} \approx \mat{P} \times \mat{Q}^T + (\mu + \vec{b} \otimes \vec{c})$$

To learn the values of $\mat{P}$ and $\mat{Q}$, we will minimize an error function:
$$E = \sum_{i,j,r_{i,j} \in T} e_{i,j} $$
$$E = \sum_{i,j,r_{i,j} \in T} \left(r_{i,j} - \hat{r}_{i,j}\right)^2 + \frac{\beta}{2} \left( ||p_i||^2 + ||q_i||^2 + b_i^2 + b_j^2 \right) $$
The first term of this error function simply describes the squared reconstruction error. The second term enforces regularization---penalizing rows of $\mat{P}$ and $\mat{Q}$ that have high magnitude. 

From this expression, we can derive $\nabla e_{i,j}$; this will allow us to minimize $e_{i,j}$ (and hence $E$) by stochastic gradient descent.

\begin{align*}
\frac{\partial e_{i,j}}{\partial p_{i,k}} &= \frac{\partial}{\partial p_{i,k}} \left(r_{i,j} - \hat{r}_{i,j}\right)^2 + \frac{\partial}{\partial p_{i,k}} \frac{\beta}{2} \left( ||p_i||^2 + ||q_i||^2 + b_i^2 + b_j^2 \right) \\
    &= 2 \left(r_{i,j} - \hat{r}_{i,j}\right)  \left(-\frac{\partial}{\partial p_{i,k}} \hat{r}_{i,j}\right)  + \frac{\partial}{\partial p_{i,k}} \frac{\beta}{2} \left( ||p_i||^2\right) \\
    &= 2 \left(r_{i,j} - \hat{r}_{i,j}\right)  \left(-\frac{\partial}{\partial p_{i,k}} \left(\mu + b_i + c_j + \vec{p_i} \cdot \vec{q_j}^T \right) \right)  + \frac{\partial}{\partial p_{i,k}} \frac{\beta}{2} \left( ||p_i||^2 \right) \\
    &= 2 \left(r_{i,j} - \hat{r}_{i,j}\right)  \left(-\frac{\partial}{\partial p_{i,k}} \left( \sum_{k=1}^{K} p_{i,k} q_{j,k} \right) \right)  + \frac{\partial}{\partial p_{i,k}} \frac{\beta}{2} \left( \sum_{k=1}^{K} p_{i,k}^2 \right) \\
    &= -2 \left(r_{i,j} - \hat{r}_{i,j}\right)  \left(q_{j,k} \right) + \beta p_{i,k} \\
\frac{\partial e_{i,j}}{\partial q_{j,k}} &= -2 \left(r_{i,j} - \hat{r}_{i,j}\right)  \left(p_{i,k} \right) + \beta q_{j,k} \\
\frac{\partial e_{i,j}}{\partial b_{i}} &= -2 \left(r_{i,j} - \hat{r}_{i,j}\right) + \beta b_{i} \\
\frac{\partial e_{i,j}}{\partial c_{j}} &= -2 \left(r_{i,j} - \hat{r}_{i,j}\right) + \beta c_{j}
\end{align*}

We can therefore implement a set of update rules as follows:
\begin{align*}
p_{i,k} &\gets p_{i,k} + \alpha \left(2 \left(r_{i,j} - \hat{r}_{i,j}\right)  q_{j,k} - \beta p_{i,k} \right) \\
q_{j,k} &\gets p_{i,k} + \alpha \left(2 \left(r_{i,j} - \hat{r}_{i,j}\right)  p_{i,k} - \beta q_{j,k} \right) \\
b_{i}   &\gets p_{i,k} + \alpha \left(2 \left(r_{i,j} - \hat{r}_{i,j}\right)          + \beta b_{i}   \right) \\
c_{j}   &\gets p_{i,k} + \alpha \left(2 \left(r_{i,j} - \hat{r}_{i,j}\right)          + \beta c_{j}   \right)
\end{align*}

\subsection{Optimization}

\subsection{Results}



\end{document}  
\begin{document}

\end{document}